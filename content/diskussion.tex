\section{Diskussion}
\label{sec:Diskussion}
Der Widerstand für die gedämpfte Messung wurde zu $R=\SI{141.62\pm1.54}{\ohm}$ bestimmt. Aus den Hersteller angaben kann für den Widerstand der Wert $R=\SI{67.2\pm0.2}{\ohm}$ entnommen werden.
Dies entspricht einer Abweichung von ungefähr $110\%$. Diese doch große Abweichung von dem Realwert lässt sich über einen Innewiderstand der Spule erklären.
Der anfallende Widerstand durch die Leitungen sollte klein genug sein um diesen zu vernachlässigen.
Der Theoriewert für den Widerstand $R_\text{ap}$ weicht um $58\%$ zu dem gemessenen Wert ab. Dies lässt sich über große Ungenauigkeiten beim Ablesen erklären, da es sehr schwierig ist zu erkennen wann der Grenzfall exakt erreicht ist.
Der Unterschied von Theorie und Messwert für die Güte der Resonanzkurve ist nur gering und liegt im Bereich $<\SI{4}{\percent}$.
Auch für die Phasenverschiebung lässt sich nur geringe Abweichung der Güte feststellen, hier sogar bei $<\SI{1}{\percent}$.
Jedoch ist zu beachten, dass für die Breite und die Phasenverschiebung durch das Ablesen der Werte eine weitere Fehlerquelle entsteht. 
