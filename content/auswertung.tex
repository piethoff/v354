\section{Auswertung}
\label{sec:Auswertung}
\subsection{Die gedämpfte Schwingung}
Die Messung wird wie in der Duchtführung beschrieben durchgeführt.
Die so erhaltenen Messwerte befinden sich in Tabelle\ref{tab:Messwerte1} :
\begin{table}[H]
    \centering
    \caption{Spannungsamplituden mit der entsprechenden Zeit.}
    \label{tab:Messwerte1}
    \begin{tabular}{S[table-format=1.1] S[table-format=2.2] }
        \toprule
        {$t/\si{\milli\second}$} & {$U_C/\si{\volt}$} \\
        \midrule
        0.000 & 13.4 \\
        0.037 & 11.4 \\
        0.076 & 9.4  \\
        0.114 & 8.2  \\
        0.152 & 7.0  \\
        0.189 & 6.0  \\
        0.228 & 5.2  \\
        0.266 & 4.4  \\
        0.304 & 3.6  \\
        0.342 & 3.0  \\
        0.379 & 2.6  \\
        0.418 & 2.2  \\
        0.456 & 2.0  \\
        0.494 & 1.8  \\
        0.532 & 1.4  \\
        \bottomrule
    \end{tabular}
\end{table}

\noindent Die Messwerte werden in der halblogarithmischen Abbildung aufgetragen.
Es wird eine lineare Ausgleichsrechnung der Form
\begin{equation*}
  f(x)=mx+b
\end{equation*}
, mit Python, durchgeführt und aufgetragen.

\begin{figure}
    \centering
    \includegraphics[width=\textwidth]{build/messung1.pdf}
    \caption{Messwerte und Ausgleichsgerade.}
    \label{fig:plot1}
\end{figure}
Es ergeben sich folgende Werte:
\begin{equation*}
  m=\SI{-4.18\pm0.05}{\per\milli\second}
\end{equation*}
\begin{equation*}
  b=2.578\pm0.017
\end{equation*}
Die Steigung ist hier
\begin{equation*}
  m=-\frac{R}{2L}.
\end{equation*}
Dies folgt aus Gleichung\eqref{eq:steig} und damit lässt sich $R$ über
\begin{equation*}
  R=-2mL
\end{equation*}
bestimmen.
Aus den Herstellerangaben findet sich für $L=\SI{16.78\pm0.09}{\milli\henry}$.
Womit
\begin{equation*}
  R=\SI{140.28\pm1.84}{\ohm}
\end{equation*}
folgt.
Die Ungenauigkeit errechnet sich hier mit der Gaußschen Fehlerfortpflanzung:
\begin{equation*}
  \Delta R= \sqrt{(\frac{\Delta m}{m})^2 + (\frac{\Delta L}{L})^2} R
\end{equation*}
Die Abklingdauer $T_\text{ex}$ ist der negative Kehrwert der Steigung $m$:
\begin{equation*}
  T_\text{ex}= -\frac{1}{m} = \SI{0.239\pm0.003}{\milli\second}
\end{equation*}


\subsection{Aperiodischer Grenzfall}
Die Messung wird wie in der Durchführung beschrieben ausgeführt.
Der so gemessene Wert für $R_\text{ap}$ beträgt:
\begin{equation*}
  R_\text{ap}=\SI{2.39}{\kilo\ohm}
\end{equation*}
Die Schwingung der Spannung verschwindet beim aperiodischen Grenzfall, daher lässt sich $R_ap$ über $L$ und $C$ bestimmen:
\begin{equation*}
  R=\sqrt{\frac{4L}{C}}
\end{equation*}
Aus den Herstellerangaben lassen sich die Werte
\begin{equation*}
  L=\SI{16.78\pm0.09}{\milli\henry}
\end{equation*}
und
\begin{equation*}
  C=\SI{2.066\pm0.006}{\nano\farad}
\end{equation*}
entnehmen.
Damit lässt sich für $R_\text{ap}$ ein Theoriewert von
\begin{equation*}
  R_\text{ap}=\SI{5.69}{\kilo\ohm}
\end{equation*}
errechnen.
