\section{Auswertung}
\label{sec:Auswertung}
\subsection{Die gedämpfte Schwingung}
\label{se:daempf}
Die Messung wird wie in der Duchtführung beschrieben durchgeführt.
Die so erhaltenen Messwerte befinden sich in Tabelle\ref{tab:Messwerte1} :
\begin{table}[H]
    \centering
    \caption{Spannungsamplituden mit der entsprechenden Zeit.}
    \label{tab:Messwerte1}
    \begin{tabular}{S[table-format=1.3] S[table-format=2.2] }
        \toprule
        {$t/\si{\milli\second}$} & {$U_C/\si{\volt}$} \\
        \midrule
        0.000 & 13.4 \\
        0.037 & 11.4 \\
        0.076 & 9.4  \\
        0.114 & 8.2  \\
        0.152 & 7.0  \\
        0.189 & 6.0  \\
        0.228 & 5.2  \\
        0.266 & 4.4  \\
        0.304 & 3.6  \\
        0.342 & 3.0  \\
        0.379 & 2.6  \\
        0.418 & 2.2  \\
        0.456 & 2.0  \\
        0.494 & 1.8  \\
        0.532 & 1.4  \\
        \bottomrule
    \end{tabular}
\end{table}

\noindent Die Messwerte werden in der halblogarithmischen Abbildung aufgetragen.
Es wird eine lineare Ausgleichsrechnung der Form
\begin{equation*}
  f(x)=mx+b
\end{equation*}
, mit Python, durchgeführt und aufgetragen.

\begin{figure}
    \centering
    \includegraphics[width=\textwidth]{build/messung1.pdf}
    \caption{Messwerte und Ausgleichsgerade.}
    \label{fig:plot1}
\end{figure}
Es ergeben sich folgende Werte:
\begin{equation*}
  m=\SI{-4.22\pm0.006}{\per\milli\second}
\end{equation*}
\begin{equation*}
  b=2.585\pm0.006
\end{equation*}
Die Steigung ist hier
\begin{equation*}
  m=-\frac{R}{2L}.
\end{equation*}
Dies folgt aus Gleichung\eqref{eq:steig} und damit lässt sich $R$ über
\begin{equation*}
  R=-2mL
\end{equation*}
bestimmen.
Aus den Herstellerangaben findet sich für $L=\SI{16.78\pm0.09}{\milli\henry}$.
Womit
\begin{equation*}
  R=\SI{141.62\pm1.54}{\ohm}
\end{equation*}
folgt.
Die Ungenauigkeit errechnet sich hier mit der Gaußschen Fehlerfortpflanzung:
\begin{equation*}
  \Delta R= \sqrt{(\frac{\Delta m}{m})^2 + (\frac{\Delta L}{L})^2} R
\end{equation*}
Die Abklingdauer $T_\text{ex}$ ist der negative Kehrwert der Steigung $m$:
\begin{equation*}
  T_\text{ex}= -\frac{1}{m} = \SI{0.237\pm0.002}{\milli\second}
\end{equation*}


\subsection{Aperiodischer Grenzfall}
Die Messung wird wie in der Durchführung beschrieben ausgeführt.
Der so gemessene Wert für $R_\text{ap}$ beträgt:
\begin{equation*}
  R_\text{ap}=\SI{2.39}{\kilo\ohm}
\end{equation*}
Die Schwingung der Spannung verschwindet beim aperiodischen Grenzfall, daher lässt sich $R_ap$ über $L$ und $C$ bestimmen:
\begin{equation*}
  R=\sqrt{\frac{4L}{C}}
\end{equation*}
Aus den Herstellerangaben lassen sich die Werte
\begin{equation*}
  L=\SI{16.78\pm0.09}{\milli\henry}
\end{equation*}
und
\begin{equation*}
  C=\SI{2.066\pm0.006}{\nano\farad}
\end{equation*}
entnehmen.
Damit lässt sich für $R_\text{ap}$ ein Theoriewert von
\begin{equation*}
  R_\text{ap}=\SI{5.69}{\kilo\ohm}
\end{equation*}
errechnen.
\subsection{Frequenzabhängigkeit des Schwingkreises}
Der Schwingkreis wird im Folgenden auf seine Frequenzabhängigkeit untersucht. Dabei entstehen die Messwerte,
welche in Tabelle \ref{tab:freqtab} aufgetragen sind.
\begin{table}[H]
        \caption{Messdaten zur Frequanzabhängigkeit.}
        \label{tab:freqtab}
        \centering
        \begin{tabular}{S[table-format=2.2(0)e0] S[table-format=1.2(0)e0] S[table-format=6.2(0)e0] }
                \toprule
                {$U_C/\si{\volt}$} & {$U/\si{\volt}$} & {$f/\si{\hertz}$} \\
                \midrule
                6.93    & 7.84  & 10.01          \\
                7.76    & 7.84  & 20.05          \\
                7.76    & 7.76  & 50.05          \\
                7.76    & 7.76  & 100.00         \\
                7.68    & 7.76  & 200.00         \\
                7.76    & 7.94  & 500.00         \\
                7.76    & 7.76  & 1000.00        \\
                7.76    & 7.76  & 2000.00        \\
                8.00    & 7.84  & 5000.00        \\
                8.87    & 7.76  & 10000.00       \\
                13.07   & 7.76  & 17500.00       \\
                14.65   & 7.76  & 19000.00       \\
                16.24   & 7.68  & 20000.00       \\
                22.97   & 7.52  & 23000.00       \\
                28.91   & 7.29  & 26000.00       \\
                17.92   & 7.52  & 30000.00       \\
                11.48   & 7.68  & 33000.00       \\
                2.81    & 7.76  & 50000.00       \\
                0.47    & 7.68  & 100000.00      \\
                \bottomrule
        \end{tabular}
\end{table}
\noindent
Diese sind in Abbildung \ref{fig:freq} aufgetragen zusammen mit einer Theoriekurve und dem theoretischen
Wert für $U_\text{max}$.
\begin{figure}[H]
    \centering
    \caption{Messwerte und Theoriekurve zur Frequenzabhängigkeit.}
    \label{fig:freq}
    \includegraphics[width=\textwidth]{build/freq.pdf}
\end{figure}
\noindent
Zusätzlich wird nochmal gesondert der Bereich um $U_\text{max}$ betrachtet, um $\omega_+$ und $\omega_-$
zu bestimmen. Dafür wurde eine Gerade mit dem Wert von $\frac{U_\text{max}}{\sqrt{2}U_0}$ eingefügt.
\begin{figure}[H]
    \centering
    \caption{Ausschnitt aus des Messwerten zur Frequenzabhängigkeit.}
    \label{fig:freq2}
    \includegraphics[width=\textwidth]{build/freq2.pdf}
\end{figure}
\noindent
Für $\omega_\pm$ ergeben sich folgende Werte:
\begin{align}
    \omega_- &=     \\
    \omega_+ &=
\end{align}
%
\subsection{Frequenzabhängigkeit des Phasenwinkels}
Im folgenden wird die Änderung des Phasenwinkels in Abhängigkeit zur Frequenz untersucht.
Der Phasenwinkel kann mithilfe folgender Formel bestimmt werden, wobei $a$ die zeitliche Verschiebung
der Signale und $b$ die Periodendauer sind:
\begin{equation}
    \phi = 2\pi \frac{a}{b}
\end{equation}
Die erhaltenen Messwerte sind in Tabelle \ref{tab:phase} zusammen mit dem berechneten Phasenwinkel
aufgetragen.
\begin{table}[H]
        \caption{Messdaten des Phasenwinkels.}
        \label{tab:phasetab}
        \centering
        \begin{tabular}{S[table-format=2.1(0)e0] S[table-format=4.2(0)e0] S[table-format=3.2(0)e0] S[table-format=1.2]}
                \toprule
                {$a/\si{\milli\second}$} & {$b/\si{\milli\second}$} & {$f/\si{\kilo\hertz}$} & {$\phi$} \\
                \midrule
                0.0     & 1000.00   & 1.00  & 0.00\\
                0.0     & 500.00    & 2.00  & 0.00\\
                1.6     & 200.08    & 5.00  & 0.05\\
                2.0     & 100.10    & 10.00 & 0.13\\
                2.4     & 66.61     & 15.00 & 0.23\\
                2.8     & 57.28     & 17.50 & 0.31\\
                3.2     & 49.97     & 20.00 & 0.40\\
                5.0     & 43.62     & 23.00 & 0.72\\
                9.4     & 38.24     & 26.13 & 1.54\\
                12.6    & 33.44     & 30.00 & 2.36\\
                12.6    & 28.90     & 33.08 & 2.74\\
                9.4     & 20.01     & 50.00 & 2.95\\
                5.0     & 9.98      & 100.00& 3.15\\
                \bottomrule
        \end{tabular}
\end{table}
\noindent
Die Messwerte sind außerdem in Abbildung \ref{fig:freq} zusammen mit einer Theoriekurve aufgetragen.
\begin{figure}[H]
    \centering
    \caption{.}
    \label{fig:freq}
    \includegraphics[width=\textwidth]{build/phase.pdf}
\end{figure}
\noindent
%
