\section{Auswertung}
\label{sec:Auswertung}
Für den Versuch wird ein Schwingkreisbaustein verwendet, bei welchem zwischen einem Widerstand
$R_1$, einem Widerstand $R_2$ und einem Zehnerpotentiometer als
Dämpfung gewählt werden kann. Die Induktivität ist angegeben mit $L$, die
Kapazität mit $C$.
\begin{table}[H]
    \centering
    \caption{Herstellerangaben des Schwingkreisbausteins.}

    \begin{tabular}{S[table-format=2.1(0.1)] S[table-format=3.0(1.0)] S[table-format=2.1(0.1)] S[table-format=1.3(1.3)] }
        \toprule
        {$R_1/\si{\ohm}$} & {$R_2/\si{\ohm}$} & {$L/\si{\milli\henry}$} &{$C/\si{\nano\farad}$} \\
        \midrule
        67.2\pm0.9 & 682\pm1 & 16.0\pm0.9 & 2.066\pm0.006
        \bottomrule
    \end{tabular}
\end{table}

\subsection{Die gedämpfte Schwingung}
\label{se:daempf}
Die so erhaltenen Messwerte befinden sich in Tabelle \ref{tab:Messwerte1} :
\begin{table}[H]
    \centering
    \caption{Spannungsamplituden mit den dazugehörigen Zeiten.}
    \label{tab:Messwerte1}
    \begin{tabular}{S[table-format=1.3] S[table-format=2.2] }
        \toprule
        {$t/\si{\milli\second}$} & {$U_C/\si{\volt}$} \\
        \midrule
        0.000 & 13.4 \\
        0.037 & 11.4 \\
        0.076 & 9.4  \\
        0.114 & 8.2  \\
        0.152 & 7.0  \\
        0.189 & 6.0  \\
        0.228 & 5.2  \\
        0.266 & 4.4  \\
        0.304 & 3.6  \\
        0.342 & 3.0  \\
        0.379 & 2.6  \\
        0.418 & 2.2  \\
        0.456 & 2.0  \\
        0.494 & 1.8  \\
        0.532 & 1.4  \\
        \bottomrule
    \end{tabular}
\end{table}

\noindent Die Messwerte werden in der halblogarithmischen Abbildung aufgetragen.
Es wird eine lineare Ausgleichsrechnung der Form
\begin{equation*}
  f(x)=mx+b,
\end{equation*}
mit Python, durchgeführt und aufgetragen.

\begin{figure}
    \centering
    \includegraphics[width=\textwidth]{build/messung1.pdf}
    \caption{Messwerte und Ausgleichsgerade.}
    \label{fig:plot1}
\end{figure}
\noindent Es ergeben sich folgende Werte:
\begin{equation*}
  m=\SI{-4.22\pm0.006}{\per\milli\second}
\end{equation*}
\begin{equation*}
  b=2.585\pm0.006
\end{equation*}
Die Steigung ist hier
\begin{equation*}
  m=-\frac{R}{2L}.
\end{equation*}
Dies folgt aus Gleichung\eqref{eq:steig} und damit lässt sich $R$ über
\begin{equation*}
  R=-2mL
\end{equation*}
bestimmen.
Aus den Herstellerangaben findet sich für $L=\SI{16.78\pm0.09}{\milli\henry}$.
Womit
\begin{equation*}
  R=\SI{141.62\pm1.54}{\ohm}
\end{equation*}
folgt.
Die Ungenauigkeit errechnet sich hier mit der Gaußschen Fehlerfortpflanzung:
\begin{equation*}
  \symup{\Delta} R= R\sqrt{\left(\frac{\Delta m}{m}\right)^2 + \left(\frac{\Delta L}{L}\right)^2}
\end{equation*}
Die Abklingdauer $T_\text{ex}$ ist der negative Kehrwert der Steigung $m$:
\begin{equation*}
  T_\text{ex}= -\frac{1}{m} = \SI{0.237\pm0.002}{\milli\second}
\end{equation*}


\subsection{Aperiodischer Grenzfall}
Der so gemessene Wert für $R_\text{ap}$ beträgt:
\begin{equation*}
  R_\text{ap}=\SI{2.39}{\kilo\ohm}
\end{equation*}
Die Schwingung der Spannung verschwindet beim aperiodischen Grenzfall, daher lässt sich $R_\text{ap}$ über $L$ und $C$ bestimmen:
\begin{equation*}
  R=\sqrt{\frac{4L}{C}}
\end{equation*}
Aus den Herstellerangaben lassen sich die Werte
\begin{equation*}
  L=\SI{16.78\pm0.09}{\milli\henry}
\end{equation*}
und
\begin{equation*}
  C=\SI{2.066\pm0.006}{\nano\farad}
\end{equation*}
entnehmen.
Damit lässt sich für $R_\text{ap}$ ein Theoriewert von
\begin{equation*}
  R_\text{ap}=\SI{5.69\pm0.017}{\kilo\ohm}
\end{equation*}
errechnen.
Die Ungenauigkeit wird hier über Gaußsche Fehlerfortpflanzung berechnet.

\subsection{Frequenzabhängigkeit des Schwingkreises}
Der Schwingkreis wird im Folgenden auf seine Frequenzabhängigkeit untersucht. Dabei entstehen die Messwerte,
welche in Tabelle \ref{tab:freqtab} aufgetragen sind.
\begin{table}
        \caption{Messdaten zur Frequenzabhängigkeit.}
        \label{tab:freqtab}
        \centering
        \begin{tabular}{S[table-format=6.2] S[table-format=2.2] S[table-format=1.2] }
                \toprule
                {$f\si{\hertz}$} & {$U_C/\si{\volt}$} & {$U/\si{\volt}$} \\
                \midrule
                10.01     & 6.93    & 7.84         \\
                20.05     & 7.76    & 7.84         \\
                50.05     & 7.76    & 7.76         \\
                100.00    & 7.76    & 7.76         \\
                200.00    & 7.68    & 7.76         \\
                300.00    & 7.76    & 7.94         \\
                1000.00   & 7.76    & 7.76         \\
                2000.00   & 7.76    & 7.76         \\
                5000.00   & 8.00    & 7.84         \\
                10000.00  & 8.87    & 7.76         \\
                17500.00  & 13.07   & 7.76         \\
                19000.00  & 14.65   & 7.76         \\
                20000.00  & 16.24   & 7.68         \\
                23000.00  & 22.97   & 7.52         \\
                26000.00  & 28.91   & 7.29         \\
                30000.00  & 17.92   & 7.52         \\
                33000.00  & 11.48   & 7.68         \\
                50000.00  & 2.81    & 7.76         \\
                100000.00 & 0.47    & 7.68         \\
                \bottomrule
        \end{tabular}
\end{table}
\noindent
Diese sind in Abbildung \ref{fig:freq} aufgetragen zusammen mit einer Theoriekurve und dem theoretischen
Wert für $U_\text{max}$ nach Gleichung \eqref{eqn:umax}.
\begin{figure}
    \centering
    \caption{Messwerte und Theoriekurve zur Frequenzabhängigkeit.}
    \label{fig:freq}
    \includegraphics[width=\textwidth]{build/freq.pdf}
\end{figure}
\noindent
Zusätzlich wird nochmal gesondert der Bereich um $U_\text{max}$ betrachtet, um $f_+$ und $f_-$
zu bestimmen. Dafür wurde eine Gerade mit dem Wert von $\frac{U_\text{max}}{\sqrt{2}U_0}$ eingefügt.
\begin{figure}[H]
    \centering
    \caption{Ausschnitt aus des Messwerten zur Frequenzabhängigkeit.}
    \label{fig:freq2}
    \includegraphics[width=\textwidth]{build/freq2.pdf}
\end{figure}
\noindent
Für $f_\pm$ ergeben sich folgende Werte:
\begin{align}
    f_- &= \SI{23050}{\hertz}    \\
    f_+ &= \SI{29000}{\hertz}
\end{align}
Somit erhält man als Breite:
\begin{equation}
    f_--f_+ = \SI{5950}{\hertz}
\end{equation}
Mit $f_0 = \SI{25900}{\hertz}$ ergibt sich eine Güte von:
\begin{equation}
    q = 4.35
\end{equation}
Im Vergleich dazu steht der Wert aus der theoretischen Überlegung nach Gleichung \eqref{eqn:breite}:
\begin{equation}
    f_{-,t}-f_{+,t} = \SI{6784}{\hertz}
\end{equation}
Es ergibt sich ein Theoriewert von:
\begin{equation}
    q_t = 4.21
\end{equation}
%
\subsection{Frequenzabhängigkeit des Phasenwinkels}
Im folgenden wird die Änderung des Phasenwinkels in Abhängigkeit zur Frequenz untersucht.
Der Phasenwinkel kann mithilfe folgender Formel bestimmt werden, wobei $a$ die zeitliche Verschiebung
der Signale und $b$ die Periodendauer sind:
\begin{equation}
    \phi = 2\pi \frac{a}{b}
\end{equation}
Die erhaltenen Messwerte sind in Tabelle \ref{tab:phase} zusammen mit dem berechneten Phasenwinkel
aufgetragen.
\begin{table}[H]
        \caption{Messdaten des Phasenwinkels.}
        \label{tab:phase}
        \centering
        \begin{tabular}{S[table-format=3.2] S[table-format=2.1(0)e0] S[table-format=4.2(0)e0] S[table-format=1.2]}
                \toprule
                {$f/\si{\kilo\hertz}$} & {$a/\si{\milli\second}$} & {$b/\si{\milli\second}$} &  {$\phi$} \\
                \midrule
                1.00   & 0.0     & 1000.00   & 0.00\\
                2.00   & 0.0     & 500.00    & 0.00\\
                5.00   & 1.6     & 200.08    & 0.05\\
                10.00  & 2.0     & 100.10    & 0.13\\
                15.00  & 2.4     & 66.61     & 0.23\\
                17.50  & 2.8     & 57.28     & 0.31\\
                20.00  & 3.2     & 49.97     & 0.40\\
                23.00  & 5.0     & 43.62     & 0.72\\
                26.13  & 9.4     & 38.24     & 1.54\\
                30.00  & 12.6    & 33.44     & 2.36\\
                33.08  & 12.6    & 28.90     & 2.74\\
                50.00  & 9.4     & 20.01     & 2.95\\
                100.00 & 5.0     & 9.98      & 3.15\\
                \bottomrule
        \end{tabular}
\end{table}
\noindent
Die Messwerte sind außerdem in Abbildung \ref{fig:phase} zusammen mit einer Theoriekurve aufgetragen.
\begin{figure}[H]
    \centering
    \caption{Messwerte zur Phasenverschiebung und Regression.}
    \label{fig:phase}
    \includegraphics[width=\textwidth]{build/phase.pdf}
\end{figure}
\noindent
Der annähernd lineare Bereich um $\phi=\pi/2$ wird genauer betrachtet.
\begin{figure}[H]
    \centering
    \caption{Ausschnitt der Messwerte zur Phasenverschiebung und Regression.}
    \label{fig:phase2}
    \includegraphics[width=\textwidth]{build/phase2.pdf}
\end{figure}
\noindent
Aus dem Diagramm erhält man folgende Werte für $\phi=\pi/2, \pi/4, 3\pi/4$:
\begin{align}
    f_0 &= \SI{26280}{\hertz} \\
    f_1 &= \SI{23320}{\hertz} \\
    f_2 &= \SI{29570}{\hertz}
\end{align}
Es folgt für die Breite:
\begin{equation}
    f_1-f_2 = \SI{6255}{\hertz}
\end{equation}
Es folgt daraus eine Güte von:
\begin{equation}
    q=4.20
\end{equation}
Im Vergleich dazu steht der Theoriewert nach Gleichung \eqref{eqn:breite}:
\begin{equation}
    f_{1,t}-f_{2,t} = \SI{6784}{\hertz}
\end{equation}
%
